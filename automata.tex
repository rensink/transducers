\section{Control automata}
\label{sec:automata}

We presuppose a universe $\Lab$ of \emph{labels}. To formally define control automata including action declaration, we have to recognise that it mediates between an \emph{abstract} and \emph{concrete} (or \emph{outside} and \emph{inside}) alphabet: if $\cC$ is such an automaton controlling $\cS$, then the concrete (inside) actions are the ones of $\cS$, whereas the abstract (outside) actions are the ones of $\cC\on \cS$. This leads us to define control automata as \emph{transducers} rather than transition systems.

\begin{definition}[control automaton]\label{def:control}
Let $X,Y\subseteq \Lambda$ be sets of actions. An \emph{$X$ to $Y$-control automaton} is a tuple $\cC=\tupof{C,L,{\trans{}},{\movesto},{\final},\tau,\qinit}$ where
\begin{itemize}[topsep=\itemsep]
\item $C$ is a set of states;
\item ${\tducex}\subseteq C\times X\times Y\times Q$ is a transduction relation;
\item ${\movesto}\subseteq C\times C$ is an \emph{or-else} relation;
\item ${\cfinal}\subseteq C$ is a termination predicate, such that $c\cfinal$ implies $c\ntducex$ and $q\nmovesto$;
\item $\tau:C\to \natN$ is a \emph{transience level}, such that $c\cfinal$ implies $\tau(c)=0$, $c\movesto c'$ implies $\tau(c')\leq \tau(c)$, and if  $\tau(c)>0$ then $(c',\x,\y,c),(c,\x',\y',c'')\in{\tducex}$ implies $\x=\x'$;
%\item ${\internal}\subseteq C$ is a set of \emph{internal states}, such that $q\internal$ implies 
%\begin{inumerate}
%\item $\tau(q)>0$ and 
%\item $q'\tduce{\x}{\dont} q\tduce{\x'}{\dont} q''$ only if $a=a'$;
%\end{inumerate}
\item $\qinit\in Q$ is the initial state, with $\tau(\qinit)=0$.
\end{itemize} 
\end{definition}
%
The set of $X$-to-$Y$ control automata is denoted $\CA_{X,Y}$. If $\tau(c)=0$, we call $c$ \emph{steady}, otherwise \emph{transient}. The additional structure with respect to an LTS can be understood as follows:
%
\begin{itemize}
\item A transduction $c\tduce\x\y c'$ invokes action $\y\in Y$ on the system while presenting it as $\x\in X$ to the outside.
\item The \emph{or-else} relation $\movesto$ has the meaning introduced in the motivation above: $c\movesto c'$ describes how control evolves if no action from $c$ is applicable.
\item The \emph{transience level} $\tau(c)$ is the depth of the atomic block nesting of $c$. Since atomic blocks are always associated with declared actions, a transient control state can only occur as internal state of the execution of such a declared action; hence every incoming and outgoing transduction will have that (abstract) action as its outer label.
%\item The \emph{internal} predicate $\internal$ holds for states that are in 
\end{itemize}
%
Though control automata and system automata are conceptually different, on a technical level it is useful to establish mappings back and forth. Any LTS $\cS\in \cS_X$ can be understood as an $X$-to-$\setof\bullet$ control automaton $\toC\cS\in \CA_{X,\setof\bullet}$ (where $\bullet$ is a fixed ``internal'' action) defined by
%
\[ \toC\cS \:\isdef\: \tupof{Q,\gensetof{q\tduce\x\bullet q'}{q\trans\x q'}, \emptyset, {\final}, \gensetof{(q,0)}{q\in Q},\qinit}
\]
%
Vice versa, a control automaton $\cC\in \CA_{X,Y}$ can be understood as a system automaton $\toS\cC$, intuitively by imagining that it is imposed on some ``universal'' one-state system automaton $\cU_Y$ with transitions $\qinit \trans\y \qinit$ for al $\y\in Y$ and $\qinit\final$. Since $\cU_Y$ does not constrain the execution of actions that $\cC$ enables, the resulting system automaton has atomic, abstract transitions for every terminating concrete transition sequence between steady states of $\cC$; and the or-else relation of $\cC$ is resolved accordingly. Formally:
%
\[ \toS\cC \:\isdef\: \tupof{\gensetof{c\in C}{\tau(c)=0},{\trans{}},{\final},\qinit} \]
%
To define $\trans{}$ and $\final$, we need an auxiliary family of \emph{reachability relations} $\setof{\Reach n}_{n\in \natN}$ with ${\Reach n}\subseteq C\times C$ for all $n\in \natN$, inductively defined by the following rules:
%
\begin{center}
\def\defaultHypSeparation{\hskip1mm}
\AxiomC{$n\geq \tau(c)\vphantom{\Reach{()}}$}
\UnaryInfC{$c\Reach n c\vphantom{c''}$}
\DisplayProof
%
\enspace
%
\AxiomC{$c\tducex c'\vphantom{\Reach{()}}$}
\AxiomC{$c'\Reach{\tau(c)} c''$}
%\AxiomC{$n < \tau(c)$}
\BinaryInfC{$c\Reach{\tau(c)} c''$}
\DisplayProof
%
\enspace
%
\AxiomC{$\neg (c\tducex \Reach{\tau(c)})$}
\AxiomC{$c\movesto c'$}
\AxiomC{$c'\Reach n c''$}
\TrinaryInfC{$c\Reach n c''$}
\DisplayProof
\end{center}
%
The resulting relation has the following, technical, property:
%
\begin{lemma}
Let $\cC$ be a control automaton. $c\Reach n c'$ if and only if is a (possibly empty) \emph{$n$-transient path} $c_0\cdots c_m$ from $c=c_0$ to $c'=c_m$, where a sequence of control states $c_0\cdots c_m$ with $m\geq 0$ is (inductively) defined to be an $n$-transient path if 
\begin{enumerate}[topsep=\itemsep]
\item For all $0\leq i<n$, either $c_i\tducex c_{i+1}$ or $c_i\movesto c_{i+1}$;
\item $\tau(c_m)\leq n$, and $\tau(c_i)>n$ for all $0\leq i<m$ (i.e., $c_m$ is the first state in the sequence with transience not exceeding $n$);
\item For $0\leq i<n$, if $c_i\movesto c_{i+1}$ then there is no $c_i\tducex c''$ such that $c''$ has an outgoing $\tau(c_i)$-transient path.
\end{enumerate}
\end{lemma}
%
Using reachability, we can now define $\trans{}$ and $\final$, being the transition relation and termination predicate of $\toS\cC$. The rules below are only considered for steady $c\in C$.
%
\begin{center}
%\def\defaultHypSeparation{\hskip1mm}
\AxiomC{$c\tduce\x\y c'$}
\AxiomC{$c'\Reach{0} c''$}
\BinaryInfC{$c\trans\x c''$}
\DisplayProof
%
\qquad
%
\AxiomC{$\neg (c\tducex \Reach{0})$}
\AxiomC{$c\movesto c'$}
\AxiomC{$c'\trans\x c''$}
\TrinaryInfC{$c\trans\x c''$}
\DisplayProof
%
\\[\medskipamount]
%
\AxiomC{$c\cfinal\vphantom{\Reach{()}}$}
\UnaryInfC{$c\final$}
\DisplayProof
%
\qquad
%
\AxiomC{$\neg (c\tducex \Reach{0})$}
\AxiomC{$c\movesto c'$}
\AxiomC{$c'\final$}
\TrinaryInfC{$c\final$}
\DisplayProof
\end{center}
%
The following is immediate and shows that $\trans{}$ and $\final$ are indeed defined over the states of $\toS\cC$, to wit the steady states of $\cC$.
%
\begin{lemma}
Let $\cC$ be a control automaton. For $c,c'\in C$, if $c\trans\x c'$ or $c'\final$, then $\tau(c')=0$.
\end{lemma}

\begin{comment}
A state with transience level $0$ is called \emph{steady} and with positive transience level \emph{transient}. We write $q\trnt[i]$ to denote $\tau(q)=i$ ($q$ has transience $i$), $q\stdy$ to denote $\tau(q)=0$ ($q$ is steady) and $q\stdy$ to denote $\tau(q)>0$ ($q$ is transient). If $\cS$ is large-step then we use $q\Tduce a\sigma q'$ to denote $(q,a,\sigma,q')\in T$, whereas if $\cS$ is small-step we use $q\tduce a\sigma q'$; in both cases, if $\sigma=\emptystr$ we may leave it out altogether. Finally, we write $q\final$ to denote $q\in Q^\Final$.

An $A$-to-$C$ transducer $\cS$ establishes a relation between abstract actions (in $A$) and (sequences of) concrete actions (in $C$). The general intuition behind a transition $(q,a,\sigma,q')$ with $q,q'\stdy$ is that this causes a sequence of actions $\sigma$ to occur, while posing to the environment that $a$ has occurred. If either $q\trnt$ or $q'\trnt$, however, then $\sigma$ is only \emph{part} of the execution of $a$; the entire sequence of concrete actions that constitutes $a$ can be found by concatenating sequences of transitions starting and ending at steady states. Importantly, action execution is considered to be \emph{atomic}, meaning that if there is no reachable steady state, the abstract action does not occur at all.

The above is captured by the operation of \emph{closure}: if $\cS$ is a small-step transducer, then its closure is the large-step transducer $\close \cS$ with state set $\close Q=Q$, initial state $\close\qinit=\qinit$, final states $\close Q^\Final=Q^\Final$ and transitions $\close T$ generated by
%
\begin{center}
\AxiomC{$q_0\tduce a{\gamma_1} q_1 \tduce a{\gamma_2} \cdots \tduce a{\gamma_n} q_n$}
\AxiomC{$q_0,q_n\stdy$}
\AxiomC{$q_1,\ldots,q_{n-1}\trnt$}
\TrinaryInfC{$q_0\Tduce a{c_1\cdots c_n} q_n$}
\DisplayProof
\end{center}
%
The following is an obvious consequence.
%
\begin{proposition}
If a small-step transducer $\cS$ is a transition system, then so is $\close{\cS}$.
\end{proposition}
%
The effect of transducers is essentially established by their \emph{composition}, $\cS_1\on \cS_2$, where $\cS_i$ for $i=1,2$ are $A_i$-to-$C_i$ transducers with $C_1=A_2$. The result is an $A_1$-to-$C_2$ transducer defined by the following rules. If the $\cS_i$ are large-step, then $\cS_1\on \cS_2$ is also a large-step transducer, defined by state set $Q=\gensetof{q_1\on q_2}{q_1\in Q_1,q_2\in Q_2}$, initial state $\qinit=\qinit_1\on \qinit_2$, and $Q^\Final$ and $T$ generated by
%
\begin{center}
\AxiomC{$q_1\Tduce a{c_1\cdots c_n} q_1'$}
\AxiomC{$q_2\Tduce{c_1}{\sigma_1}\cdots\Tduce{c_n}{\sigma_n} q_2'$}
\BinaryInfC{$q_1\on q_2 \Tduce a{\sigma_1\cdots \sigma_n} q_1'\on q_2'$}
\DisplayProof
%
\qquad
%
\AxiomC{$q_1\final$}
\AxiomC{$q_2\final$}
\BinaryInfC{$q_1\on q_2\;\final$}
\DisplayProof
\end{center}
%
For small-step transducers, composition is similar, but includes a rule for the transience level:
\end{comment}
%
Due to the mappings $\rS$ and $\rC$, when introducing new concepts we can start out by defining them for control automata; their meaning for system automata follows. For instance, we will prove our results modulo bisimilarity, which is a relation over control automata defined as follows.

\begin{definition}[bisimilarity]\label{def:bisimilarity}
Let $\cC_1,\cC_2\in \CA_{X,Y}$ be control automata. $\cC_1$ and $\cC_2$ are \emph{bisimilar}, denoted $\cC_1\sim \cC_2$, if there exists a binary relation ${\bis}\subseteq C_1\times C_2$ such that $\qinit_1\bis\qinit_2$, and if $c_1\bis c_2$ then
\begin{itemize}
\item $c_1\tduce \x\y c_1'$ [$c_2\tduce \x\y c_2'$] implies $\exists c_2'\st c_2\tduce \x\y c_2'$ [$\exists c_1'\st c_1\tduce \x\y c_1'$] such that $c_1'\bis c_2'$;
\item $c_1\movesto c_1'$ [$c_1\movesto c_1'$] implies $\exists c_2'\st c_2\movesto c_2'$ [$\exists c_1'\st c_2\movesto c_1'$] such that $c_1'\bis c_2'$;
\item $\tau_1(c_1)=\tau_2(c_2)$;
\item $c_1\cfinal_1$ if and only if $c_1\cfinal_2$.
\end{itemize}
\end{definition}
%
For system automata $\cS_1,\cS_2\in \SA_{X}$, we define $\cS_1\sim \cS_2$ iff $\toC{\cS_1}=\toC{\cS_2}$. (Note that this comes down to the standard notion over LTSs.) The following states that $\rC$ and $\rS$ are well-defined modulo bisimilarity, and that $\rS$ is left inverse to $\rC$. The proof is straightforward and omitted here.

\begin{proposition}
Let $\cS_1,\cS_2\in \SA_X$ and $\cC_1,\cC_2\in \CA_{X,Y}$ such that $\cS_1\sim \cS_2$ and $\cC_1\sim \cC_2$. 
\begin{align*}
\toC{\cS_1} & \sim \toC{\cS_2} \\
\toS{\cC_1} & \sim \toS{\cC_2} \\
\toS{\toC{\cS_1}} & \sim \cS_1 \enspace.
\end{align*}
\end{proposition}

\paragraph{Imposition.}

Introduced in \Cref{sec:motivation}, $\cC\on\cS$ is the prime construction of this paper, in which the control automaton $\cC$ restricts the behaviour of the system automaton $\cS$, resulting in another system automaton. True to the principle formulated above, though, we define this as an operation on control automata, $\cC_1\on \cC_2$, from which the imposition on system automata is derived.

For $\cC_1\in \CA_{X,Y}$ and $\cC_2\in \CA_{Y,Z}$, $\cC=\cC_1\on \cC_2\in \CA_{X,Z}$ is given by state set $C=\gensetof{c_1\on c_2}{c_1\in C_1, c_2\in C_2}$, initial state $\qinit=\qinit_1\on \qinit_2$, transience function $\tau:c_1\on c_2 \mapsto \tau(c_1)+\tau(c_2)$, and transduction, or-else and termination defined by the following rules:

\begin{center}
\insertBetweenHyps{\hskip 3pt}
\AxiomC{$c_1\tduce\x\y c_1'$}
\AxiomC{$c_2\tduce\y\z c_2'$}
\BinaryInfC{$c_1\on c_2 \tduce\x\z c_1'\on c_2'$}
\DisplayProof
%
\,
%
\AxiomC{$c_1\movesto c_1'$}
\UnaryInfC{$c_1\on c_2 \movesto c_1'\on c_2$}
\DisplayProof
%
\,
%
\AxiomC{$c_2\movesto c_2'$}
\UnaryInfC{$c_1\on c_2 \movesto c_1\on c_2'$}
\DisplayProof
%
\,
%
\insertBetweenHyps{\hskip 3pt}
\AxiomC{$c_1\cfinal$}
\AxiomC{$c_2\cfinal$}
\BinaryInfC{$c_1\on c_2\;\cfinal$}
\DisplayProof
\end{center}
%
It is straightforward to show that this is well-defined modulo bisimilarity (if $\cC_1\sim \cC_1'$ and $\cC_2\sim \cC_2'$ then $\cC_1\on \cC_2\sim \cC_1'\on \cC_2'$). Furthermore, for arbitrary $X$, if $\cI_X\in \CA_{X,X}$ is the \emph{identity control automaton}, with singleton state set $C=\setof\qinit$, $\qinit\cfinal$, ${\movesto}=\emptyset$ and transductions $\qinit\tduce \x\x \qinit$ for all $\x\in X$, then the following equivalences hold for arbitrary $\cC_1\in\CA_{W,X}$, $\cC_2\in \CA_{X,Y}$ and $\cC_3\in \CA_{Y,Z}$.
%
\begin{align*}
\cI_{W}\on \cC_1 & \sim \cC_1 \\
\cC_1\on \cI_X & \sim \cC_1 \\
\cC_1\on(\cC_2\on \cC_3) & \sim (\cC_1\on \cC_2)\on \cC_3 \enspace.
\end{align*}
%
Imposition on system automata, is then given (for $\cC\in \CA_{X,Y}$ and $\cS\in \SA_Y$) by:
%
\[ \cC\on \cS \:\isdef\: \toS{\cC\on \toC\cS} \]
%
Much less trivially, but quite importantly for the consistency of our setup, is the following result stating that the conversion to a system automaton commutes with imposition.

\begin{proposition}
Let $\cC_1\in \CA_{X,Y}$ and $\cC_2\in \CA_{Y,Z}$; then $\cC_1\on \toS{\cC_2} \sim \toS{\cC_1\on \cC_2}$.
\end{proposition}
%
For any set of actions $A\subseteq \Lambda$, the \emph{identity transducer} $\cI_A$ is a trivial, one-state $A$-to-$A$ transducer, where the one state (say $q$) is both initial and final, and there are transitions from $q$ to $q$ for all $a\in A$ that merely convert $a$ into itself. Formally, this is given by $\cI_A=\tupof{\setof q,q,T,\setof q,\setof{(q,0)}}$ for some arbitrary $q$, where $T$ is generated by
%
\begin{center}
\AxiomC{$a\in A$}
\UnaryInfC{$q\tduce a a q$}
\DisplayProof
\end{center}
%
Note that $\cI_A$ is both large-step and small-step. The identity transducer acts as the identity of composition, both on large-step and on small-step transducers. Moreover, the composition of small-step transducers is consistent with that of their closure.
%
\begin{proposition}
Let $\cS_i$ be $A_i$-to-$C_i$ transducers for $i=1,2,3$, with $A_i=C_{i+1}$ for $i=1,2$. The following equalities hold, if all $\cS_i$ are either large-step or small-step:
\begin{align*}
\cS_1\on(\cS_2\on \cS_3) & \equiv (\cS_1\on \cS_2)\on \cS_3 \\
\cS_1\on \cI_{C_1} & \equiv \cS_1 \\
\cI_{A_1}\on \cS_1 & \equiv \cS_2 \enspace.
\end{align*}
If $\cS_1$ and $\cS_2$ are small-step, then in addition the following holds:
\begin{align*}
\close \cS_1\on \close \cS_2 & \equiv \closex{\cS_1\on \cS_2} \enspace.
\end{align*}
Finally, if $\cS_2$ is a transition system, then so is $\cS_1\on \cS_2$.
\end{proposition}
%
A \emph{large-step $A$-to-$C$ refinement function} is a function $R:A\to \powersetof{C^+}$ for some sets $A,C\subseteq \Lab$ of abstract, respectively concrete actions. This essentially implements all abstract actions as sequences of concrete actions: a given abstract $a\in A$ is implemented as a choice of any non-empty sequence of concrete actions $\sigma\in R(a)$. The effect of $R$ is captured by the single-state large-step transducer $\cR_R=\tupof{\setof p,p,T,\setof p,\setof{(p,0)}}$, where $p$ is some arbitrary fresh state and $T$ is defined by
%
\begin{center}
\AxiomC{$a\in A$}
\AxiomC{$\sigma\in R(a)$}
\BinaryInfC{$p\Tduce a\sigma p$}
\DisplayProof
\end{center}
%
Given a large-step $A$-to-$C$ refinement function $R$, we can now either \emph{refine} a $D$-to-$A$ transducer $\cS$ by applying it to $\cR_R$ (as in $\cS\on \cR_R$) or \emph{abstract} a $C$-to-$D$ transducer $\cS$ by applying $\cR_R$ to it (as in $\cR_R\on \cS$).

\medskip\noindent 
Next, we set out to mimic, or implement, action refinement for small-step transducers. Rather than to a set of non-empty traces, refinement now maps every abstract action to a small-step transducer. Formally, this is captured by a \emph{small-step $A$-to-$C$ refinement function}, which is a function $r:A\to \STduce(\any,C)$ such that all $r(a)$ are non-empty and unambiguously terminating. Below we use $\cS_a$ to denote $r(a)$, and $P_a=\setof{\qinit_a}\cup Q^\Final_a$ to denote the set of initial and final states of $\cS_a$

To capture the effect of a small-step refinement function $r$, we again turn it into a transducer. To define this formally, without loss of generality we assume all $Q_a$ to be disjoint. The small-step $A$-to-$C$-transducer $\cR_r$ has state set $Q =\setof p\cup\bigcup_{a\in A} (Q_a\setminus P_a)$ for some fresh state $p$, and initial state $\qinit=p$. Let $f_r$ map all $q\in P_a$ to $p$ and all $q\in Q_a\setminus P_a$ to themselves (for all $a\in A$); then the other components of $\cR_r$ are generated by the following rules:
%
\begin{center}
\AxiomC{$q\tduce[a] b \gamma q'$}
\UnaryInfC{$f_r(q)\tduce a \gamma f_r(q')$}
\DisplayProof
%
\quad
%
\AxiomC{$\vphantom{\final_r}$}
\UnaryInfC{$p\final$}
\DisplayProof
%
\quad
%
\AxiomC{$\vphantom{\final_r}$}
\UnaryInfC{$p\trnt0\vphantom{\final}$}
\DisplayProof
\end{center}
%
The compatibility of small-step to large-step refinement takes the following shape. For a small-step transducer $\cS$, let $\Traces(\cS)=\gensetof{\gamma_1\cdots\gamma_n}{\qinit\tduce {a_1}{\gamma_1}\cdots \tduce{a_n}{\gamma_n} q'\final}$ denote the \emph{concrete traces} of $\cS$. For a small-step $A$-to-$C$ refinement function $r$, the closure of $r$ is a large-step refinement function $\close r$ defined as
%
\[ \close r:a\mapsto \Traces(r(a)) \quad\text{for all $a\in A$} \enspace. \]
%
We then have the following correspondence:
%
\begin{proposition}
If $r$ is a small-step $A$-to-$C$ refinement function, then $\close r$ is a large-step $A$-to-$C$ refinement function such that $\cR_r\equiv \cR_{\close r}$.
\end{proposition}
%
Small-step refinement also satisfies a useful distributivity property. First let us extend transducer composition to small-step refinement functions by defining, for an $A$-to-$C$ refinement function $r$ and a $C$-to-$D$ transducer $\cS$:
%
\[ r\on \cS: a \mapsto r(a)\on S \quad \text{for all $a\in A$} \]
%
Clearly, $r$ is an $A$-to-$E$ refinement function.
%
\begin{proposition}
Let $r$ be a small-step $A$-to-$C$ refinement function and $\cS$ a small-step $C$-to-$E$ transducer. The following equation holds:
\[ \cR_{r\on \cS} \equiv \cR_{r}\on \cS \enspace. \]
\end{proposition}
