\section{Control automata}
\label{sec:automata}

We presuppose a universe $\Lab$ of \emph{labels}. For subsets $A\subseteq \Lambda$, we use $A^\emptystr$ to denote $A\cup\setof\emptystr$, where $\emptystr\notin\Lambda$ is a special symbol signifying the empty string.

\begin{definition}[transducer]
Let $A,C\subseteq \Lambda$ be sets of actions. An \emph{$A$ to $C$-transducer} is a tuple $\cS=\tupof{Q,\qinit,Q^\Stdy,Q^\Final,T}$ where
\begin{itemize}[topsep=\itemsep]
\item $Q$ is a set of states;
\item $\qinit\in Q^\Stdy$ is the initial state;
\item $T\subseteq Q\times A\times C^*\times Q$ is a transition relation, such that $(q,a,\sigma,q'),(q',b,\rho,q'')\in T$ with $q'\notin Q^\Stdy$ implies $a=b$.
\item $Q^\Final\subseteq Q^\Stdy$ is the subset of \emph{final} states;
\item $\tau:Q\to \natN$ is a \emph{transience level}, such that $(q,a,\sigma,q'),(q',b,\rho,q'')\in T$ with $\tau(q')>0$ implies $a=b$.
\end{itemize}
A transducer $\cS$ is called
\begin{itemize}
\item \emph{large-step} if $\tau(q)=0$ for all $q\in Q$,
\item \emph{small-step} if $(q,a,\sigma,q')\in T$ implies $\sigma\in C^\emptystr$,
\item \emph{pure} if $(q,a,\sigma,q')\in T$ implies $\sigma\in C$, and 
\item a \emph{transition system} if $C=\emptyset$ (hence $(q,a,\sigma,q')\in T$ implies $\sigma=\emptystr$).
\item \emph{empty} if $\qinit \in Q^\Final$.
\item \emph{unambiguously terminating} if $(q,a,\sigma,q')$ implies $q\notin Q^\Final$.
\end{itemize}
Two transducers $\cS_1,\cS_2$ are \emph{isomorphic}, denoted $\cS_1\equiv \cS_2$, if there is a bijective mapping $f:Q_1\to Q_2$ such that $f(\qinit_2)=\qinit_2$, $f(Q^\Final_1)=Q^\Final_2$, $f(T_1)=T_2$ (where in the last two equations, $f$ is tacitly extended pointwise), and $\tau_1=\tau_2\circ f$.
\end{definition}
%
The class of large-step $A$-to-$C$-transducers is denoted $\LTduce(A,C)$, that of small-step $A$-to-$C$ transducers $\STduce(A,C)$. We will also use $\STduce(\any,C)$ to denote the union of $\STduce(A,C)$ for arbitrary $A\subseteq\Lab$.

A state with transience level $0$ is called \emph{steady} and with positive transience level \emph{transient}. We write $q\trnt[i]$ to denote $\tau(q)=i$ ($q$ has transience $i$), $q\stdy$ to denote $\tau(q)=0$ ($q$ is steady) and $q\stdy$ to denote $\tau(q)>0$ ($q$ is transient). If $\cS$ is large-step then we use $q\Tduce a\sigma q'$ to denote $(q,a,\sigma,q')\in T$, whereas if $\cS$ is small-step we use $q\tduce a\sigma q'$; in both cases, if $\sigma=\emptystr$ we may leave it out altogether. Finally, we write $q\final$ to denote $q\in Q^\Final$.

An $A$-to-$C$ transducer $\cS$ establishes a relation between abstract actions (in $A$) and (sequences of) concrete actions (in $C$). The general intuition behind a transition $(q,a,\sigma,q')$ with $q,q'\stdy$ is that this causes a sequence of actions $\sigma$ to occur, while posing to the environment that $a$ has occurred. If either $q\trnt$ or $q'\trnt$, however, then $\sigma$ is only \emph{part} of the execution of $a$; the entire sequence of concrete actions that constitutes $a$ can be found by concatenating sequences of transitions starting and ending at steady states. Importantly, action execution is considered to be \emph{atomic}, meaning that if there is no reachable steady state, the abstract action does not occur at all.

The above is captured by the operation of \emph{closure}: if $\cS$ is a small-step transducer, then its closure is the large-step transducer $\close \cS$ with state set $\close Q=Q$, initial state $\close\qinit=\qinit$, final states $\close Q^\Final=Q^\Final$ and transitions $\close T$ generated by
%
\begin{center}
\AxiomC{$q_0\tduce a{\gamma_1} q_1 \tduce a{\gamma_2} \cdots \tduce a{\gamma_n} q_n$}
\AxiomC{$q_0,q_n\stdy$}
\AxiomC{$q_1,\ldots,q_{n-1}\trnt$}
\TrinaryInfC{$q_0\Tduce a{c_1\cdots c_n} q_n$}
\DisplayProof
\end{center}
%
The following is an obvious consequence.
%
\begin{proposition}
If a small-step transducer $\cS$ is a transition system, then so is $\close{\cS}$.
\end{proposition}
%
The effect of transducers is essentially established by their \emph{composition}, $\cS_1\on \cS_2$, where $\cS_i$ for $i=1,2$ are $A_i$-to-$C_i$ transducers with $C_1=A_2$. The result is an $A_1$-to-$C_2$ transducer defined by the following rules. If the $\cS_i$ are large-step, then $\cS_1\on \cS_2$ is also a large-step transducer, defined by state set $Q=\gensetof{q_1\on q_2}{q_1\in Q_1,q_2\in Q_2}$, initial state $\qinit=\qinit_1\on \qinit_2$, and $Q^\Final$ and $T$ generated by
%
\begin{center}
\AxiomC{$q_1\Tduce a{c_1\cdots c_n} q_1'$}
\AxiomC{$q_2\Tduce{c_1}{\sigma_1}\cdots\Tduce{c_n}{\sigma_n} q_2'$}
\BinaryInfC{$q_1\on q_2 \Tduce a{\sigma_1\cdots \sigma_n} q_1'\on q_2'$}
\DisplayProof
%
\qquad
%
\AxiomC{$q_1\final$}
\AxiomC{$q_2\final$}
\BinaryInfC{$q_1\on q_2\;\final$}
\DisplayProof
\end{center}
%
For small-step transducers, composition is similar, but includes a rule for the transience level:
%
\begin{center}
\insertBetweenHyps{\hskip 3pt}
\AxiomC{$q_1\tduce a c q_1'$}
\AxiomC{$q_2\tduce c \gamma q_2'$}
\BinaryInfC{$q_1\on q_2 \tduce a \gamma q_1'\on q_2'$}
\DisplayProof
%
\,
%
\AxiomC{$q_1\tduce a \emptystr q_1'$}
\UnaryInfC{$q_1\on q_2 \tduce a \emptystr q_1'\on q_2$}
\DisplayProof
%
\,
%
\insertBetweenHyps{\hskip 3pt}
\AxiomC{$q_1\final$}
\AxiomC{$q_2\final$}
\BinaryInfC{$q_1\on q_2\;\final$}
\DisplayProof
%
\,
%
\insertBetweenHyps{\hskip 3pt}
\AxiomC{$q_1\trnt[i]$}
\AxiomC{$q_2\trnt[j]$}
\BinaryInfC{$q_1\on q_2\;\trnt[i+j]$}
\DisplayProof
\end{center}
%
For any set of actions $A\subseteq \Lambda$, the \emph{identity transducer} $\cI_A$ is a trivial, one-state $A$-to-$A$ transducer, where the one state (say $q$) is both initial and final, and there are transitions from $q$ to $q$ for all $a\in A$ that merely convert $a$ into itself. Formally, this is given by $\cI_A=\tupof{\setof q,q,T,\setof q,\setof{(q,0)}}$ for some arbitrary $q$, where $T$ is generated by
%
\begin{center}
\AxiomC{$a\in A$}
\UnaryInfC{$q\tduce a a q$}
\DisplayProof
\end{center}
%
Note that $\cI_A$ is both large-step and small-step. The identity transducer acts as the identity of composition, both on large-step and on small-step transducers. Moreover, the composition of small-step transducers is consistent with that of their closure.
%
\begin{proposition}
Let $\cS_i$ be $A_i$-to-$C_i$ transducers for $i=1,2,3$, with $A_i=C_{i+1}$ for $i=1,2$. The following equalities hold, if all $\cS_i$ are either large-step or small-step:
\begin{align*}
\cS_1\on(\cS_2\on \cS_3) & \equiv (\cS_1\on \cS_2)\on \cS_3 \\
\cS_1\on \cI_{C_1} & \equiv \cS_1 \\
\cI_{A_1}\on \cS_1 & \equiv \cS_2 \enspace.
\end{align*}
If $\cS_1$ and $\cS_2$ are small-step, then in addition the following holds:
\begin{align*}
\close \cS_1\on \close \cS_2 & \equiv \closex{\cS_1\on \cS_2} \enspace.
\end{align*}
Finally, if $\cS_2$ is a transition system, then so is $\cS_1\on \cS_2$.
\end{proposition}
%
A \emph{large-step $A$-to-$C$ refinement function} is a function $R:A\to \powersetof{C^+}$ for some sets $A,C\subseteq \Lab$ of abstract, respectively concrete actions. This essentially implements all abstract actions as sequences of concrete actions: a given abstract $a\in A$ is implemented as a choice of any non-empty sequence of concrete actions $\sigma\in R(a)$. The effect of $R$ is captured by the single-state large-step transducer $\cR_R=\tupof{\setof p,p,T,\setof p,\setof{(p,0)}}$, where $p$ is some arbitrary fresh state and $T$ is defined by
%
\begin{center}
\AxiomC{$a\in A$}
\AxiomC{$\sigma\in R(a)$}
\BinaryInfC{$p\Tduce a\sigma p$}
\DisplayProof
\end{center}
%
Given a large-step $A$-to-$C$ refinement function $R$, we can now either \emph{refine} a $D$-to-$A$ transducer $\cS$ by applying it to $\cR_R$ (as in $\cS\on \cR_R$) or \emph{abstract} a $C$-to-$D$ transducer $\cS$ by applying $\cR_R$ to it (as in $\cR_R\on \cS$).

\medskip\noindent 
Next, we set out to mimic, or implement, action refinement for small-step transducers. Rather than to a set of non-empty traces, refinement now maps every abstract action to a small-step transducer. Formally, this is captured by a \emph{small-step $A$-to-$C$ refinement function}, which is a function $r:A\to \STduce(\any,C)$ such that all $r(a)$ are non-empty and unambiguously terminating. Below we use $\cS_a$ to denote $r(a)$, and $P_a=\setof{\qinit_a}\cup Q^\Final_a$ to denote the set of initial and final states of $\cS_a$

To capture the effect of a small-step refinement function $r$, we again turn it into a transducer. To define this formally, without loss of generality we assume all $Q_a$ to be disjoint. The small-step $A$-to-$C$-transducer $\cR_r$ has state set $Q =\setof p\cup\bigcup_{a\in A} (Q_a\setminus P_a)$ for some fresh state $p$, and initial state $\qinit=p$. Let $f_r$ map all $q\in P_a$ to $p$ and all $q\in Q_a\setminus P_a$ to themselves (for all $a\in A$); then the other components of $\cR_r$ are generated by the following rules:
%
\begin{center}
\AxiomC{$q\tduce[a] b \gamma q'$}
\UnaryInfC{$f_r(q)\tduce a \gamma f_r(q')$}
\DisplayProof
%
\quad
%
\AxiomC{$\vphantom{\final_r}$}
\UnaryInfC{$p\final$}
\DisplayProof
%
\quad
%
\AxiomC{$\vphantom{\final_r}$}
\UnaryInfC{$p\trnt0\vphantom{\final}$}
\DisplayProof
\end{center}
%
The compatibility of small-step to large-step refinement takes the following shape. For a small-step transducer $\cS$, let $\Traces(\cS)=\gensetof{\gamma_1\cdots\gamma_n}{\qinit\tduce {a_1}{\gamma_1}\cdots \tduce{a_n}{\gamma_n} q'\final}$ denote the \emph{concrete traces} of $\cS$. For a small-step $A$-to-$C$ refinement function $r$, the closure of $r$ is a large-step refinement function $\close r$ defined as
%
\[ \close r:a\mapsto \Traces(r(a)) \quad\text{for all $a\in A$} \enspace. \]
%
We then have the following correspondence:
%
\begin{proposition}
If $r$ is a small-step $A$-to-$C$ refinement function, then $\close r$ is a large-step $A$-to-$C$ refinement function such that $\cR_r\equiv \cR_{\close r}$.
\end{proposition}
%
Small-step refinement also satisfies a useful distributivity property. First let us extend transducer composition to small-step refinement functions by defining, for an $A$-to-$C$ refinement function $r$ and a $C$-to-$D$ transducer $\cS$:
%
\[ r\on \cS: a \mapsto r(a)\on S \quad \text{for all $a\in A$} \]
%
Clearly, $r$ is an $A$-to-$E$ refinement function.
%
\begin{proposition}
Let $r$ be a small-step $A$-to-$C$ refinement function and $\cS$ a small-step $C$-to-$E$ transducer. The following equation holds:
\[ \cR_{r\on \cS} \equiv \cR_{r}\on \cS \enspace. \]
\end{proposition}
