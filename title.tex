\title{A control calculus with priority choice and atomic composition}

\author{Arend Rensink}

\begin{abstract}
If we have a model that generates a state space, it is often desirable to restrict the fragment of the state space that is actually considered. If transitions are labelled, a well-known way to achieve this is through a \emph{control program} that specifies the transitions of interest from each state.

The contribution of this paper is to present a process-algebraic semantics for a control calculus with two powerful features, namely \emph{priority choice} and \emph{atomic composition} (a variant of \emph{action refinement}). Where solutions for the first of these have been studied before, the combination with the second necessitates the extension of the control automaton model with transducer-based capabilities. The functionality of the resulting calculus subsumes that of existing control languages, for instance in rule-based systems.
\end{abstract}
