\title{A control calculus with priority choice and atomic composition}
\author{Arend Rensink}
\maketitle

\begin{abstract}
Starting with a model of which the behaviour is given in the form of a state space, we often want to concentrate on only a fragment of that behaviour. If transitions are labelled, a well-known way to achieve this is through a \emph{control program} (sometimes also called a \emph{strategy}) that restricts the transitions of interest from each state.

The contribution of this paper is to present a process-algebraic semantics for a control calculus with two powerful features, namely \emph{priority choice} and \emph{atomic composition} (a variant of \emph{action refinement}). Whereas solutions for both of these have been studied before, their combination has not, and raises non-trivial issues that we address here. The functionality of the resulting calculus subsumes that of existing control languages, for instance in rule-based systems.
\end{abstract}
