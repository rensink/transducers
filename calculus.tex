\section{A calculus for control}
\label{sec:calculus}

In the following, the nonterminal $\rC$ stands for arbitrary control programs. The nonterminal $\rR$ generates the fragment of non-empty, finite programs. Finally, $\rF$ stands for a refinement function: $a_1$ up to $a_n$ should be thought of as abstract actions, which are mapped to $\rR$-terms over concrete actions.
%
\begin{align*}
\rC & \ruleis a \rulemid \term \rulemid \rC_1;\rC_2 \rulemid \rC_1+\rC_2 \rulemid \rC_1\orelse \rC_2 \rulemid \alap \rC \color{gray} \rulemid \rC[\rF] \rulemid a@\rC \color{black} \\
\rR & \ruleis a \rulemid \rR_1;\rC_2 \rulemid \rR_1+\rR_2 \rulemid \rR_1\orelse \rR_2 \rulemid \rR[\rF] \\
\rF & \ruleis a_1 \mapsto \rR_1, \cdots, a_n\mapsto \rR_n \enspace.
\end{align*}
%
Non-standard operators are:
\begin{itemize}
\item \emph{Priority choice} $\rC_1\orelse \rC_2$. Tries to execute $\rC_1$; if this fails, tries to execute $\rC_2$.
\item \emph{As-long-as-possible} $\alap \rC$. Can be thought of as equivalent to $\rC;\alap \rC\orelse \term$.
\item \emph{Refinement} $\rC[\rF]$. When $\rC$ invokes action $a$, $\rC[\rF]$ atomically invokes $\rF(a)$.
\item \emph{Intermediate state $a@\rC$}. Reached during the atomic execution of a refined $a$.
\end{itemize}
%
We will explain the purpose and functioning of control programs by first restricting to the fragment without refinement; i.e., the black clauses of $\rC$. For the semantics, we adapt the approach of \cite{CCS-Priority-Choice} to the setting of this paper. (See \Cref{sec:related} for a brief discussion of the adaptations.)

A control program $\rC$ is \emph{imposed} on an existing \emph{target system} $\cS$, the behaviour of which is given by a labelled transition system (LTS); the composed system is denoted $\rC\on\cS$, or (during execution) $\rC\on q$ for some state $q$ of $\cS$. In the composed system, $\rC$ \emph{invokes} actions of the target system in a given order and $q$ \emph{executes} them; however, not all actions that $\rC$ tries to invoke are necessarily applicable (executable) in $q$, and if not, the invocation fails. These roles (invocation and execution) roughly correspond to \emph{output} (by the control program) versus \emph{input} (of the LTS), as used in CCS \cite{CCS}.

In a control program of the form $\rC_1\orelse \rC_2$, the operand $\rC_2$ is ``activated'' if and only if none of the actions initially invoked by $\rC_1$ are executable in the target system. The control program $\alap \rC_1$ terminates in that case. Hence action invocation by, as well as termination of, a control program is in general \emph{conditional} on the failure of other, previously attempted actions. This effect is captured by the following predicates:
%
\begin{itemize}
\item $\rC \tries X$ for some set of actions $X$, denoting that the control program $\rC$ can only progress by invoking one of the actions in $X$ (and in particular, cannot immediately terminate);
\item $\rC\trans{a\ifnot X} \rC'$ (read: ``$a$ unless $X$''), denoting that the control program $\rC$ invokes $a$, evolving into $\rC'$, \emph{unless} one of the actions in $X$ is executable. As we will see, this is only used in case $\rC\tries X$, so the actions occurring as negative conditions for the invocation of $a$ have actually themselves been tried.
\item $q \cando X$ for some set of actions $X$, denoting that at least one of the actions in $X$ is executable in the current state $q$;
\end{itemize}
%
Each of these predicates can be defined using SOS-type rules. Two crucial rules are:
%
\begin{center}
\AxiomC{$\rC_1\tries X$}
\AxiomC{$\rC_2\trans{a\ifnot Y} \rC'$}
\BinaryInfC{$\rC_1\orelse \rC_2\displaytrans{a\ifnot(X\cup Y)} \rC'$}
\DisplayProof
%
\quad
%
\AxiomC{$\rC\trans{a\ifnot X} \rC'$}
\AxiomC{$\neg (q\cando X)$}
\AxiomC{$q\trans a q'$}
\TrinaryInfC{$\rC\on q \displaytrans a \rC'\on q'$}
\DisplayProof
\end{center}





Terms and refinement functions are \emph{typed}: $\rC:A,C$ (respectively $\rF:A,C$) denotes that $\rC$ (respectively $\rF$) has abstract alphabet $A$ and concrete alphabet $C$. We give the typing rules for $\rC$ and $\rF$; those for $\rR$ coincide with the ones for $\rC$.
%
\begin{center}
\AxiomC{$a\in A\cap C\vphantom{\rC_1}$}
\UnaryInfC{$a:A,C$}
\DisplayProof
%
\;
%
\AxiomC{$\vphantom{\rC_1}$}
\UnaryInfC{$\term:A,C$}
\DisplayProof
%
\;
%
\AxiomC{$\rC_1:A,C$}
\AxiomC{$\rC_2:A,C$}
\BinaryInfC{$\begin{array}[t]{c}
\rC_1;\rC_2:A,C \\
\rC_1+\rC_2:A,C \\
\rC_1\orelse\rC_2:A,C \\
\alap\rC_1:A,C
\end{array}$}
\DisplayProof
%
\\[\medskipamount]
%
\AxiomC{$\rC:A,B$}
\AxiomC{$\rF:B,C$}
\BinaryInfC{$\rC[\rF]:A,C$}
\DisplayProof
%
\;
%
\AxiomC{$a\in A$}
\AxiomC{$\rC:A,C$}
\BinaryInfC{$a@\rC:A,C$}
\DisplayProof
%
\\[\medskipamount]
%
\AxiomC{$a_1,\ldots,a_n\in A$}
\AxiomC{$A\setminus\setof{a_1,\ldots,a_n}\subseteq C$}
\AxiomC{$\rR_1:B_1,C \quad\cdots\quad \rR_n:B_n,C$}
\TrinaryInfC{$\rF:A,C$}
\DisplayProof
\end{center}
%
Note that the type of a term is not unique. Below, we use $\rF:A,C$ as a function from $A$ to terms $\rF(a):\any,C$, mapping actions not explicitly listed to themselves. If $\rF\equiv a_1 \mapsto \rR_1, \cdots, a_n\mapsto \rR_n$, then for arbitrary $a\in A$ this function is defined by
%
\[ \rF:a\mapsto \begin{cases}
                \rR_i & \text{if $a=a_i$} \\
                a & \text{otherwise.}
                \end{cases} \]
%
The small-step semantics of our calculus is determined by operational rules. Transitions are generated by the following rules:
%
\begin{center}
\AxiomC{$\vphantom\final$}
\UnaryInfC{$a \tduce a a \term\vphantom\final$}
\DisplayProof
%
\;
%
\AxiomC{$\rC_1\tduce a c \rC'$}
\UnaryInfC{$\rC_1;\rC_2\tduce a c \rC';\rC_2$}
\DisplayProof
%
\;
%
\AxiomC{$\rC_1\final$}
\AxiomC{$\rC_2\tduce a c \rC'$}
\BinaryInfC{$\rC_1;\rC_2\tduce a c\rC'$}
\DisplayProof
%
\\[\medskipamount]
%
\AxiomC{$\rC_1\tduce a c \rC'$}
\UnaryInfC{$\rC_1+\rC_2\tduce a c \rC'$}
\DisplayProof
%
\;
%
\AxiomC{$\rC_2\tduce a c \rC'$}
\UnaryInfC{$\rC_1+\rC_2\tduce a c \rC'$}
\DisplayProof
%
\\[\medskipamount]
%
\AxiomC{$\rC\tduce a c \rC'$}
\AxiomC{$\rF(c)\tduce b e \rC''$}
\BinaryInfC{$\rC[\rF]\tduce a e a@\rC'';\rC'[\rF]$}
\DisplayProof
%
\;
%
\AxiomC{$\rC\tduce b c \rC'$}
\UnaryInfC{$a@\rC\tduce a c a@\rC'$}
\DisplayProof
\end{center}
%
Termination is generated by the following rules:
%
\begin{center}
\AxiomC{$\vphantom\final$}
\UnaryInfC{$\term\final$}
\DisplayProof
%
\;
%
\AxiomC{$\rC_1\final$}
\AxiomC{$\rC_2\final$}
\BinaryInfC{$\rC_1;\rC_2\;\final$}
\DisplayProof
%
\;
%
\AxiomC{$\rC_1\final$}
\UnaryInfC{$\rC_1+\rC_2\;\final$}
\DisplayProof
%
\;
%
\AxiomC{$\rC_2\final$}
\UnaryInfC{$\rC_1+\rC_2\;\final$}
\DisplayProof
%
\;
%
\AxiomC{$\rC\final$}
\UnaryInfC{$\rC[\rF]\final$}
\DisplayProof
%
\;
%
\AxiomC{$\rC\final$}
\UnaryInfC{$a@\rC\final$}
\DisplayProof
\end{center}
%
The rules for the transience level are:
%
\begin{center}
\def\defaultHypSeparation{\hskip .0in}
\AxiomC{$\vphantom\final$}
\UnaryInfC{$a\trnt[0]\vphantom\rC$}
\DisplayProof
%
\,
%
\AxiomC{$\vphantom\final\vphantom\final$}
\UnaryInfC{$\term\trnt[0]\vphantom\rC$}
\DisplayProof
%
\,
%
\AxiomC{$\rC_1\trnt[i]\vphantom\final$}
\AxiomC{$\rC_2\trnt[j]$}
\BinaryInfC{$\rC_1;\rC_2\;\trnt[i]$}
\DisplayProof
%
\,
%
\AxiomC{$\rC_1\trnt[i]\vphantom\final$}
\AxiomC{$\rC_2\trnt[j]$}
\BinaryInfC{$\rC_1+\rC_2\;\trnt[0]$}
\DisplayProof
%
\,
%
\AxiomC{$\rC\trnt[i]\vphantom\final$}
\UnaryInfC{$\rC[\rF]\trnt[i]$}
\DisplayProof
%
\,
%
\AxiomC{$\rC\final$}
\UnaryInfC{$a@\rC\trnt[0]$}
\DisplayProof
%
\,
%
\AxiomC{$\rC\trnt[i]$}
\AxiomC{$\neg \rC\final$}
\BinaryInfC{$a@\rC\trnt[i+1]$}
\DisplayProof
\end{center}
%
The rules for sequential composition and choice are noteworthy: they essentially ignore the transience of the second operant, respectively both operands. This is a simplification justified by the fact that reachable terms will always have steady operands at those positions. 