\documentclass{llncs}
\usepackage{preamble}
\begin{document}
We presuppose a universe $\Lab$ of \emph{labels}. For subsets $A\subseteq \Lambda$, we use $A^\emptystr$ to denote $A\cup\setof\emptystr$, where $\emptystr\notin\Lambda$ is a special symbol signifying the empty string.

\begin{definition}[transducer]
Let $A,C\subseteq \Lambda$ be sets of actions. An \emph{$A$ to $C$-transducer} is a tuple $S=\tupof{Q,Q^\Stdy,Q^\Final,\qinit,T}$ where
\begin{itemize}[topsep=\itemsep]
\item $Q$ is a set of states;
\item $Q^\Stdy\subseteq Q$ is the subset of \emph{steady} states;
\item $Q^\Final\subseteq Q^\Stdy$ is the subset of \emph{final} states;
\item $\qinit\in Q^\Stdy$ is the initial state;
\item $T\subseteq (Q\setminus Q^\Final)\times A\times C^*\times Q$ is a transition relation, such that $(q,a,\sigma,q'),(q',b,\rho,q'')\in T$ with $q'\notin Q^\Stdy$ implies $a=b$.
\end{itemize}
A transducer is
\begin{itemize}
\item \emph{large-step} if $Q^\Stdy=Q$,
\item \emph{small-step} if $(q,a,\sigma,q')\in T$ implies $\sigma\in C^\emptystr$,
\item \emph{pure} if $(q,a,\sigma,q')\in T$ implies $\sigma\in C$, and 
\item a \emph{transition system} if $Q^\Stdy=Q$ and $C=\emptyset$ (hence $(q,a,\sigma,q')\in T$ implies $\sigma=\emptystr$).
\end{itemize}
\end{definition}
%
A state that is not steady is called \emph{transient}; we use $Q^\Trnt=Q\setminus Q^\Stdy$ to denote the set of transient states. If $S$ is large-step then we use $q\Tduce a\sigma q'$ to denote $(q,a,\sigma,q')\in T$, whereas if $S$ is small-step we use $q\tduce a\sigma q'$; in both cases, if $\sigma=\emptystr$ we typically leave it out altogether. Finally, we write $q\stdy$ to denote $q\in Q^\Stdy$, $q\trnt$ to denote $q\in Q^\Trnt$ and $q\final$ to denote $q\in Q^\Final$.

An $A$-to-$C$ transducer $S$ establishes a relation between abstract actions (in $A$) and (sequences) of concrete actions (in $C$). The general intuition behind a transition $(q,a,\sigma,q')$ with $q,q'\in Q^\Stdy$ is that this causes a sequence of actions $\sigma$ to occur, while posing to the environment that $a$ has occurred. If either $q$ or $q'$ is transient, however, then $\sigma$ is only \emph{part} of the execution of $a$; the entire sequence of concrete actions that constitutes $a$ can be found by concatenating sequences of transitions starting and ending at steady states. That is, if $S$ is a small-step transducer, then there is a corresponding large-step transducer $\bar S$ with the same state set $Q$, final states $Q^\Final$, initial state $\qinit$, but $\bar T$ generated by
%
\begin{center}
\AxiomC{$q_0\tduce a{c_1} q_1 \tduce a{c_2} \cdots \tduce a{c_n} q_n$}
\AxiomC{$q_0,q_n\stdy$}
\AxiomC{$q_1,\ldots,q_{n-1}\trnt$}
\TrinaryInfC{$q_1\Tduce a{c_1\cdots c_n} q_n$}
\DisplayProof
\end{center}

The effect of transducers is essentially established by their \emph{composition}, $S_1\on S_2$, where $S_i$ for $i=1,2$ are $A_i$-to-$C_i$ transducers with $C_1=A_2$. The result is an $A_1$-to-$C_2$ transducer defined by the following rules. If the $S_i$ are large-step, then $S_1\on S_2$ is also a large-step transducer, defined by state set $Q=\gensetof{q_1\on q_2}{q_1\in Q_1,q_2\in Q_2}$, $Q^\Stdy=\emptyset$, $\qinit=\qinit_1\on \qinit_2$ and $Q^\Final$ and $T$ generated by
%
\begin{center}
\AxiomC{$q_1\Tduce a{c_1\cdots c_n} q_1'$}
\AxiomC{$q_2\Tduce{c_1}{\sigma_1}\cdots\Tduce{c_n}{\sigma_n} q_2'$}
\BinaryInfC{$q_1\on q_2 \Tduce a{\sigma_1\cdots \sigma_n} q_1'\on q_2'$}
\DisplayProof
%
\qquad
%
\AxiomC{$q_1\final$}
\AxiomC{$q_2\final$}
\BinaryInfC{$q_1\on q_2\;\final$}
\DisplayProof
\end{center}
%
For small-step transducers, the composition is similar, but also needs a rule for stability:
%
\begin{center}
\AxiomC{$q_1\tduce a c q_1'$}
\AxiomC{$q_2\tduce c e q_2'$}
\BinaryInfC{$q_1\on q_2 \Tduce a e q_1'\on q_2'$}
\DisplayProof
%
\quad
%
\AxiomC{$q_1\tduce a \emptystr q_1'$}
\UnaryInfC{$q_1\on q_2 \Tduce a \emptystr q_1'\on q_2$}
\DisplayProof
%
\quad
%
\AxiomC{$q_1\stdy$}
\AxiomC{$q_2\stdy$}
\BinaryInfC{$q_1\on q_2\;\stdy$}
\DisplayProof
%
\quad
%
\AxiomC{$q_1\final$}
\AxiomC{$q_2\final$}
\BinaryInfC{$q_1\on q_2\;\final$}
\DisplayProof
\end{center}


\end{document}